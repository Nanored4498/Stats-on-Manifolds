\chapauthor{Yoann Coudert--Osmont}

\motto{
	Ce chapitre a pour but de définir un modèle pour les données longitudinales vivant sur une variété.
}

\chapter{Modèles à effets mixtes pour données longitudinales}
\label{chap:longitudinal}

\section{Données longitudinales}

\begin{figure}[b]
	\centering
	\begin{tikzpicture}
		\draw[->] (-0.1, 0) -- (6, 0) node[right] {$t$};
		\draw[->] (0, -0.1) -- (0, 2.6) node[left] {$y$};
		
		\node[blue] at (0.5, 1.2) {$\times$};
		\node[blue] at (1.3, 1.7) {$\times$};
		\node[blue] at (2.1, 2.15) {$\times$};
		\draw[orange] (-0.2, 0.8) -- (2.7, 2.5) node[right] {$\gamma_1$};
		\node[blue] at (1.6, 0.8) {$\times$};
		\node[blue] at (2.25, 1.25) {$\times$};
		\node[blue] at (3, 1.9) {$\times$};
		\draw[orange] (0.4, -0.1) -- (3.8, 2.5) node[right] {$\gamma_2$};
		\node[blue] at (2.8, 0.6) {$\times$};
		\node[blue] at (3.7, 0.9) {$\times$};
		\node[blue] at (4.1, 1.6) {$\times$};
		\draw[orange] (2.15, -0.1) -- (5.4, 2.5) node[right] {$\gamma_3$};
		\node[blue] at (3.9, 0.3) {$\times$};
		\node[blue] at (5, 1) {$\times$};
		\draw[orange] (3.27, -0.1) -- (5.8, 1.51) node[right] {$\gamma_4$};
		\draw[red, thick] (1, -0.1) -- (4.8, 2.5) node[right] {$\gamma_0$};
		
		\draw[greenTikz, dotted] (-0.2, 2.15) -- (5.8, 0.3) node[right] {\small regression sur toutes les données};
	\end{tikzpicture}
	\caption{Régression dans un modèle longitudinale.}
\end{figure}

Dans un modèle longitudinale, on a plusieurs sujets au nombre de $N$. Pour chaque sujet $i \in \{ 1, ..., N \}$, on observe $N_i$ données $\{ y_{i,j} \}_{1 \leqslant j \leqslant N_i}$ à des temps $\{ t_{i,j} \}_{1 \leqslant j \leqslant N_i}$. Au lieu de faire une régression comme vu dans le chapitre précédent (\ref{sec:Regression}) sur toutes les données d'un coup on peut faire une régression sur chaque sujet :
\begin{equation}
	\cancel{y_{i,j} = a t_{i,j} + b + \epsilon_{i,j}} \qquad y_{i,j} = a_i t_{i,j} + b_i + \epsilon_{i,j}
\end{equation}
On suppose de plus que les paramètres de notre régression sont distribués selon des gaussiennes :
\begin{equation}
	a_i \sim \mathcal{N}(a, \sigma_a^2) \qquad b_i \sim \mathcal{N}(b, \sigma_b^2) \qquad \epsilon_{i,j} \sim \mathcal{N}(0, \sigma_\epsilon^2)
\end{equation}
On introduit de nouvelles variables $\tilde{a}_i$ et $\tilde{b}_i$ défini par :
\begin{equation}
	\tilde{a}_i = a_i - a \qquad \tilde{b}_i = b_i - b
\end{equation}
Ce qui nous permet ensuite de définir une trajectoire moyenne $\gamma_0$ ainsi que des trajectoires $\gamma_i$ pour chaque sujet $i$ par :
\begin{equation}
	\gamma_0(t) = at + b \qquad \text{et} \qquad \gamma_i(t) = \gamma_0(t) + \tilde{a}_it + \tilde{b}_i
\end{equation}
Dans le cas d'une variété on prendre à nouveau pour $\gamma_0$ une géodésique partant de $p_0$ avec une vitesse $v_0$ :
\begin{equation}
	\gamma_0(t) = \Exp_{p_0}(t v_0)
\end{equation}
Mais $\gamma_i$ est défini comme une somme de la trajectoire moyenne $\gamma_0$ et d'une autre petite trajectoire. Il n'existe pas d'équivalent immédiat sur les variétés. Comme on s'attend à ce que les trajectoires des différents sujets soient en quelque sorte parallèles, on peut vouloir définir une manière de construire des géodésiques parallèles sur une variété, pour obtenir des trajectoires $\gamma_i$ en effectuant des translations de géodésiques parallèlement à $\gamma_0$. On va donc avoir besoin de définir des opérateurs $T_i$ sur les trajectoires tels que :
\begin{equation}
	\gamma_i(t) = T_i(\gamma_0)(t)
\end{equation}
On voudra aussi que notre opérateur $T$ soit compatible avec les distorsions temporelles. C'est à dire que pour $\psi : \R \rightarrow \R$, $\mathcal{C}^1$ croissante on a l'égalité :
\begin{equation}
	T(\gamma_0 \circ \psi) = T(\gamma_0) \circ \psi
\end{equation} 

\section{Exp-parallélisation, Modèle à effets mixtes}

\subsection{Exp-parallélisation}

\begin{definition}[Exp-parallélisation]
	Soit $\gamma_0 : I \subset \R \rightarrow \M$ une courbe différentiable sur $\M$ une variété différentiable géodésiquement complète. Pour $t_0 \in I$ et $w \in T_{\gamma_0(t_0)} \M$, l'Exp-parallèle à $\gamma_0$ dans la direction $w$ est la courbe :
	\begin{equation}
		\eta_{\gamma_0}^{t_0, w}(t) = \Exp_{\gamma_0(t)} \left( P_{\gamma_0}^{t_0, t}(w) \right)
	\end{equation}
	Où $P_{\gamma_0}^{t_0, t}(w)$ est le transport parallèle de $w$ le long de $\gamma_0$ de $t_0$ à $t$. C'est à dire la solution de :
	\begin{equation}
		\syst{lll}{
			\nabla_{\dot{\gamma}_0} P_{\gamma_0}^{t_0, t}(w) = 0 \\[2mm]
			P_{\gamma_0}^{t_0, t_0}(w) = w
		}
	\end{equation}
\end{definition}

\begin{proposition}
	Soit $\gamma_0 : I \rightarrow \M$ une courbe différentiable sur $\M$, $t_0 \in I$ et $w \in T_{\gamma_0(t_0)} \M$. Alors $\eta_{\gamma_0}^{t_0, w}$ est une courbe différentiable et :
	\begin{description}
		\item[$\bullet$] $\displaystyle \eta_{\gamma_0}^{t_0, w} = \eta_{\gamma_0}^{t_0', P_{\gamma_0}^{t_0, t_0'}(w)}$ \quad Cette opération est intrinsèque
		\item[$\bullet$] Si $\psi$ est une fonction strictement croissante de $\R$ tel que $I \subset \psi(\R)$ alors $\displaystyle \eta_{\gamma_0 \circ \psi}^{\psi^{-1}(t_0), w} = \eta_{\gamma_0}^{t_0, w} \circ \psi$. \quad Cette opération est compatible aux distorsions temporelles.
	\end{description}
\end{proposition}

\begin{figure}[b]
\centering
\begin{tikzpicture}[scale=0.7]
\shade[left color=darkWhite!90, right color=whiteGray!90] 
(1.5,-2) to[out=15,in=170] (11.2,-1.8) node[above right] {\large $\M$} to[out=85, in=-70] (10.2,2.5) to[out=165,in=15] (1,2.2) -- cycle;

\coordinate (p0) at (2.3, -0.8);
\coordinate (p1) at (9.6, 1);
\coordinate (p4) at (1.7, 0.6);
\coordinate (p5) at (9, 2.4);

\draw[red, thick] (p0) to[out=28, in=172] node[below,pos=0.8] {$\gamma_0(t)$} coordinate[pos=0.12] (p2) coordinate[pos=0.65] (p3) (p1);
\draw[blue, thick] (p4) to[out=28, in=172] node[below,pos=0.9] {$\eta_{\gamma_0}^{t_0,w}(t)$} coordinate[pos=0.14] (p6) coordinate[pos=0.72] (p7) (p5);
\draw[greenTikz] (p2){}+(0.6, 1) --+ (-1.2, 0.7) --+ (-0.9, -1.05) --+ (1, -0.7) node[right] {$T_{p_0} \M$} -- cycle;

\draw[->, thick] (p2) --+ ({1.1*cos(110)}, {1.1*sin(110)});
\draw (p2){}+(-0.5, 0.4) node {$w$};
\draw[->, thick] (p3) --+ ({1.1*cos(98)}, {1.1*sin(98)});
\draw (p3){}+(-0.9, 0.4) node {$P_{\gamma_0}^{t_0,t}(w)$};
\draw[red] (p2) node {$\bullet$} node[below] {$p_0$};

\draw[dashed, thick] (p2) to[out=110, in=-80] (p6);
\draw[dashed, thick] (p3) to[out=98, in=-95] (p7);
\draw[blue] (p6) node {$\bullet$} node[above=3pt] {$\Exp_{p_0}(w)$};
\end{tikzpicture}
\caption{Exp-parallèle d'une courbe différentiable.}
\end{figure}

\begin{proof}
	\begin{description}
		\item[$\bullet$] On commence par appliqué la définition de l'Exp-parallèle aux deux côtés de l'égalité :
		\begin{equation}
			\eta_{\gamma_0}^{t_0, w}(t) = \Exp_{\gamma_0(t)} \left( P_{\gamma_0}^{t_0, t}(w) \right) \qquad \quad \eta_{\gamma_0}^{t_0', P_{\gamma_0}^{t_0, t_0'}(w)}(t) = \Exp_{\gamma_0(t)} \left( P_{\gamma_0}^{t_0', t} \circ P_{\gamma_0}^{t_0, t_0'}(w) \right)
		\end{equation}
		On pose alors deux champs de vecteurs parallèles à $\gamma_0$, $f$ et $g$ suivant :
		\begin{equation}
			f(t) = P_{\gamma_0}^{t_0, t}(w) \qquad \qquad g(t) = P_{\gamma_0}^{t_0', t} \circ P_{\gamma_0}^{t_0, t_0'}(w)
		\end{equation}
		On remarque ensuite que $f$ et $g$ sont tous deux solutions d'une même équation à solution unique :
		\begin{equation}
			\nabla_{\dot{\gamma_0}} f(t) = \nabla_{\dot{\gamma_0}} g(t) = 0 \qquad \qquad f(t) = g(t) = w
		\end{equation}
		Donc $f = g$ ce qui montre le premier point.
		
		\item[$\bullet$] On refait de même pour le second point :
		\begin{equation}
			\eta_{\gamma_0 \circ \psi}^{\psi^{-1}(t_0), w}(t) = \Exp_{\gamma_0 \circ \psi(t)} \left( P_{\gamma_0 \circ \psi}^{\psi^{-1}(t_0), t}(w) \right) \qquad \quad \eta_{\gamma_0}^{t_0, w} \circ \psi(t) = \Exp_{\gamma_0 \circ \psi(t)} \left( P_{\gamma_0}^{t_0, \psi(t)}(w) \right)
		\end{equation}
		On pose à nouveau les deux champs de vecteurs parallèles à $\gamma_0$, $f$ et $g$ suivant :
		\begin{equation}
			f(t) = P_{\gamma_0 \circ \psi}^{\psi^{-1}(t_0), t}(w) \qquad \qquad g(t) = P_{\gamma_0}^{t_0, \psi(t)}(w)
		\end{equation}
		Par définition on a :
		\begin{equation}
			\nabla_{\dot{\gamma_0 \circ \psi}} f(t) = 0 \qquad
		\end{equation}
		Pour $g$ il faut utiliser la connexion d'une composition :
		\begin{equation}
			\nabla_{\dot{\gamma_0 \circ \psi}} g(t) = \dot{\psi}(t) \left( \nabla_{\dot{\gamma}_0} g \right) \circ \psi(t)
		\end{equation}
		Or on sait que $\nabla_{\dot{\gamma}_0} g = 0$, d'où :
		\begin{equation}
			\nabla_{\dot{\gamma_0 \circ \psi}} g(t) = 0
		\end{equation}
		De plus on a les égalités suivantes :
		\begin{equation}
			f \circ \psi^{-1}(t_0) = g \circ \psi^{-1}(t_0) = w
		\end{equation}
		Étant solutions de la même équation, $f$ et $g$ sont donc égaux et on en déduit ce point.
	\end{description}
	\qed
\end{proof}

\begin{proposition}
	Soit $\gamma_0(t) = \Exp_{p_0}((t - t0) v_0)$ la géodésique qui passe au point $p_0$ au temps $t_0$ avec la vitesse $v_0$. Soit la fonction de distorsion temporelle $\phi(t) = \alpha (t - t_0') + t_0$. Alors $\tilde{\gamma} = \gamma \circ \psi$ est la géodésique qui passe par $p_0$ au temps $t_0'$ avec une vitesse $\alpha v_0$.
\end{proposition}

\begin{proof}
	Vérifions que $\tilde{\gamma}$ est bien une géodésique :
	\begin{equation}
		\nabla_{\dot{\tilde{\gamma}}} \dot{\tilde{\gamma}}(t) = \dot{\psi}(t) \left( \nabla_{\dot{\tilde{\gamma}}} \dot{\gamma} \right) \circ \psi(t) = \dot{\psi}^2(t) \left( \nabla_{\dot{\gamma}} \dot{\gamma} \right) \circ \psi(t) = 0
	\end{equation}
	La dernière égalité venant du fait que $\gamma$ est une géodésique. Ensuite en remarquant que $\psi(t_0') = t_0$ on a :
	\begin{equation}
		\tilde{\gamma}(t_0') = \gamma(t_0) = p_0 \qquad \dot{\tilde{\gamma}}(t_0') = \dot{\psi}(t_0') \dot{\gamma(t_0)} = \alpha v_0
	\end{equation}
	\qed
\end{proof}

\begin{remarque}
	\label{rk:ortho_vw}
	Si on prend une géodésique $\gamma_0$ de vitesse initiale $v_0$ et $w = \alpha v_0$, on obtient :
	\begin{equation}
		\eta_{\gamma_0}^{t_0, w}(t) = \Exp_{\gamma_0(t)} \left( \alpha P_{\gamma_0}^{t_0,t}(v_0) \right) = \Exp_{\gamma_0(t)}(\alpha \dot{\gamma_0}(t))) = \gamma_0 \left( (\alpha + 1) t \right)
	\end{equation}
	Donc l'Exp parallèle dans la direction de la géodésique revient à une distorsion temporelle. Comme on va utiliser une paire composé d'une translation par Exp-parallélisation et d'une distorsion temporelle dans notre modèle, si on veut assurer l'unicité de cette paire il nous faut choisir une translation orthogonale à $v_0$. C'est à dire :
	\begin{equation}
		\label{eq:ortho_vw}
		g_{p_0}(v_0, w) = 0
	\end{equation}
\end{remarque}

\subsection{Modèle à effets mixtes}

Maintenant que nous avons les outils pour, on modélise nos données par :
\begin{equation}
	y_{i, j} = \eta_{\gamma_0}^{t_0, w_i} \circ \psi_i(t_{i,j}) + \epsilon_{i,j} \qquad \text{avec } \; \gamma_0(t) = \Exp_{p_0} \left( (t - t_0) v_0 \right)
\end{equation}
D'après la remarque précédente \ref{rk:ortho_vw}, on doit avoir l'orthogonalité entre les $w_i$ et $v_0$. On pourra représenter ces vecteurs dans une base $\left( A_k \right)_{1 \leqslant k < d}$ déterminé à l'avance :
\begin{equation}
	w_i = A s_i = \sum_{k=1}^{d-1} A_k s_{i,k} \qquad \text{avec } \; g_{p_0}(v_0, A_k) = 0
\end{equation}
On prendra une distorsion temporelle affine qui permet au sujet de $i$ (en oubliant la translation) de passer par $p_0$ au temps $t_0 + \tau_i$, avec une vitesse $\alpha_i v_0$ :
\begin{equation}
	\psi_i(t) = \alpha_i \left( t - t_0 - \tau_i \right) + t_0
\end{equation}
On écrira $\alpha_i = e^{\xi_i}$. Puis toutes ces variables sont modéliser statistiquement par les lois suivantes :
\begin{equation}
	\tau_i \sim \mathcal{N}(0, \sigma_\tau^2) \qquad \xi_i \sim \mathcal{N}(0, \sigma_\xi^2) \qquad s_{i,k} \sim \mathcal{N}(0, 1) \qquad \epsilon_{i,j} \sim \mathcal{N}(0, \sigma_\epsilon^2 I_d)
\end{equation}